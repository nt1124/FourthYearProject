\documentclass[a4paper,10pt]{article}
\usepackage[utf8]{inputenc}
\usepackage{color}
\usepackage{graphicx}

\usepackage{amsmath}
\usepackage{amsfonts}

\usepackage[margin=1.4in]{geometry}
\setcounter{tocdepth}{2}


\newcommand{\HRule}{\rule{\linewidth}{0.5mm}}


\begin{document}

	\begin{titlepage}
		\begin{center}

			\textsc{\LARGE University of Bristol}
			\vspace{2cm}
    
			\includegraphics[width=0.15\textwidth]{./bristolunilogo}~\\[1cm]
    
			\textsc{\Large Background Chapter}
    
			\vspace{3cm}
			\HRule \\[0.4cm]
			\textsc{ \huge \bfseries Secure Two Party Computation}
			\HRule \\[0.4cm]
			\vspace{2cm}
    
			\begin{abstract}
				We shall be producing an implementation of the Secure Two Party Computation protocol using Cut-and-choose put forward by Prof. Lindell in \cite{Lindell_CnC_2014}. We shall also implement other previous protocols that are also secure against Malicious and Covert adversaries for the purposes of comparison.
			\end{abstract}
    
			\vspace{3cm}
			\noindent\rule{12cm}{0.2pt}\\[0.2cm]
			\textsc{\Large Author: Nicholas Tutte (nt1124)}\\[0.2cm]
			\textsc{\Large Supervisor: Prof. Nigel Smart}
    
		\end{center}
	\end{titlepage}

	\section{Background Chapter}
		\subsection{Problem definition} \label{InformalProbDef}
			Secure multi-party computation(SMC) is a fundamental problem in Cryptography. We have a set of parties who wish to cooperate to compute some function on inputs distributed across the parties. However, these parties distrust one another and do not wish their inputs to be known to the other parties. We shall be focusing on the case where there are only two parties(S2PC), but most two party approaches can be generalised to the multi-party case.\\

			A commonly used example is the Millionaires problem. A group of rich persons wish to find out who among them is the richest, but do not wish to tell each other how much they are worth. Here the parties are the rich (and somewhat vain) individuals. Their inputs are their net worths and the function will return the identifier of the individual with the highest input. Finally no party should be able to divine anything about anothers inputs, apart from what the parties can infer from their own input and the output.\\
      
			Whilst this is not exactly an inspiring application, it does explain convey the problem concisely. We shall cover further applications later in \ref{Applications}.\\

			Throughout we will assume that we can establish a secure and authenticated channel of communication to all other parties in the computation. That is we assume communications between two parties cannot be eavesdropped upon or altered, and that we can detect attempts to impersonate another party.\\

			Some time should be taken at this point to note also that we are pursuing a generalised solution to this problem. Many SMC protocols exist that are designed for a specific problem, whilst these are often more efficient than any general solution they are limited to one problem. We will be focusing on general protocols that can be applied to any problem.


		\subsection{Formal ideal model}
			There are three main properties that we wish to achieve with any SMP protocol,
			\begin{itemize}
				\item Privacy, the only knowledge parties gain from participating is the output.
				\item Correctness, the output is indeed that of the intended function.
				\item Independence of inputs, no party can choose it's inputs as the function of other parties inputs.
			\end{itemize}

			In this sense we define the goal of an adversary to compromise any one of these properties.\\

			We compare any protocol to the \emph{ideal} execution, in which the parties submit their inputs to a universally trusted and incorruptible external party via secure channels. This trusted party then computes the value of the function and returns the output to the parties. We say that the protocol is secure if no adversary can attack the protocol with more success than they can achieve than the ideal model.


		\subsection{Security levels}\label{sub:securityLevels}
			Having established what goals the adversary wishes to achieve and how we can measure if said adversary has a valid attack, we next deal with the question of the capabilities of the adversary. We use three main models to describe the threat level of the adversary.

			\subsubsection{Semi-honest Adversary}
				The Semi Honest(SH) adversary is the weakest adversary, and requires the strongest limitations on their capabilities. The SH adversary has also been referred to as ``honest but curious'', because in this case the adversary is not allowed to deviate from the established protocol in any way (i.e. they play fair/are honest), but at the same time they will do their best to compromise one of the aforementioned security properties by examining the data they have legitimate access to. This is in some ways analogous to the classic ``passive'' adversary.

			\subsubsection{Malicious Model}
				The Malicious adversary is allowed to employ any polynomial time strategy and is not bounded by the protocol (they can run arbitrary code instead), furthermore the Malciious adversary does not care if it is caught cheating so long as it achieves its goal in the process. This is in some ways analogous to the classic ``active'' adversary.

			\subsubsection{Covert Model}
				The Covert adversary model is very similar to the Malicious model, again bounded by polynomial time with freedom to ignore the protocol, however in this case the adversary is adverse to being caught cheating. A Covert Adversary will accept a certain known probability of detection, this probability represents the point at which the expected payofff/benefit of cheating without detection outweighs the expected cost/punishment for getting caught.\\

				We call the probability that a Covert adversary will be caught the ``deterrent probability'', usually denoted using $\epsilon$. Often protocols providing security against Covert adversaries take a Security parameter which varies the probability of detecting cheating.


		\subsection{Applications} \label{Applications}
			At first it might appear that SMC lacks applications beyond those like the trivial example provided in \ref{InformalProbDef}. In fact as this is often the first example given many dismiss SMC as of limited usefulness or as a pointless toy. Here we shall provide a number of other applications either already in use or in development.


			\subsubsection{Secret Auctions - Danish Beets} \label{BeetsAuctionApplication}
				In Denmark a significant number of farmers are contracted to grow sugar beets for Danisco (a Danish bio-products company). Farmers can trade contracts amongst themselves (effectively sub-contracting the production of the beets), bidding for these sub-contracts is done via a ``double auction''.\\

				Farmers do not wish to expose their bids as this gives information about their financial state to Danisco and so refused to accept Danisco as a trusted auctioneer, similarly all other parties (e.g. Farmer union) already involved are in some way disqualified. Rather than rely on a completely uninvolved party like an external auction house the farmers use an SMC-based approached described in \cite{SugarBeets}. Since 2008 this auction has been run multiple times \\

				As far as team behind this auction are aware this is the first large scale application of SMC to a real world problem, this application example in particular is important as it is a concrete practical example of SMC being used to solve a problem demonstrating this is not just a Cryptological gimmick.

			\subsubsection{Database queries} \label{LegalDatabaseApplication}
				Imagine the case where one party holds a database, and the other wishes to perform a query upon this database. However, for whatever reason the querying party does not wish to reveal what their query is, nor does the holder of the database wish to give the database to the querier. This particular problem has attracted significant attention in the academic community and also in the legal domain, as it provides a way for parties to comply with legal requirements to give up information without having to provide unfetted access to external parties.

			\subsubsection{Distributed secrets} \label{DistributedSecretApplication}
				Consider the growing use of physical tokens in user authentication, e.g. the RSA SecurID. When each SecurID token is activated the seed generated for that token is loaded to the relevant server (RSA Authentication Manager), then when authentication is needed both the server and the token compute `something' using the aforementioned seed. However, this means that in the event of the server being breached and the seed being compromised the physical tokens will need to be replaced. Clearly this is undesirable, being both expensive both in terms of up front clean up costs and reputation.\\

				In the above scenario we clearly need to store the secret(the seed) somewhere, but if we can split the seed across multiple servers and then get the servers to perform the computation as a SMC problem (where each server's input is their share of the secret, the output the value to compare to the token's input) then we can increase the cost to an attacker, as they will now have to compromise multiple servers. Such a service is in development by Dyadic Security (full disclosure, my supervisor Prof. Nigel Smart is a co-founder of Dyadic).

			\subsubsection{AES Encryption} \label{AES_Application}
				A classic test/benchmark for any general S2PC protocol is to perform an AES encryption where the message and key are held by two parties, so $P_A$ can input a message to be encrypted, $P_B$ inputs a key, and the protocol returns the encrypted ciphertext to $P_A$ and nothing to $P_B$. At no point did $P_A$ know what key was used, nor $P_B$ what the plaintext was.


		\subsection{Yao Garbled Circuits} \label{Yao_Circuits}

			\subsubsection{Overview} \label{Yao_Overview}
				Yao garbled circuits are one of the primary avenues of research into Secure multi-party computation. Yao first proposed garbled circuits \cite{YaoOriginal}. The two parties are designated the Builder and the Executor. The Builder then constructs a circuit representing the function the parties wish to compute, this circuit is ``garbled'' in such a way that it can stil be executed.\\

				This garbled circuit, hardcoded with the Builders input data, is sent to the Executing party who then obtains the data representing their input form the Builder via Oblivious Transfer. The Executor then evaluates the circuit and obtains the output of the function.


			\subsubsection{Details} \label{Yao_Details}
				As noted above we first represent the function to compute as a binary circuit. Denote the two parties as $P_1$ and $P_2$, we will assume WLOG that $P_1$ is building the circuit whilst $P_2$ is executing. Now take a single gate of this circuit with two input wires and a single output wire. Denote the gate a $G_1$ and the input wires as $w_1$ and $w_2$, let $b_i \in \{0, 1\}$ be the value of $w_i$. Here we will take the case where $w_i$ is an input wire for which $P_i$ provides the value. Define the output value of the gate to be $G(b_1, b_2) \in \{0, 1\}$. We now garble this gate in order to obscure the inputs and outputs.\\

				$P_1$ (the circuit building party) garbles each wire by selecting two random keys of length $l$, for the wire $w_i$ call these keys $k_i^0$ and $k_i^1$. The length of these keys ($l$) can be considered a security parameter, and should correspond to the length of the key needed for the symmetric encryption scheme we'll be using later. Further $P_1$ also generates a random permuation $\pi_i \in \{0, 1\}$ for each $w_i$. We define $c_i = \pi_i(b_i)$. The garbled value of the $i^{th}$ wire is then $k_i^{b_i} \Vert c_i$, we then represent our garbled truth table for the gate with the table indexed by the values for the $c_1$ and $c_2$. We extend the truth table this to a gate with more than two inputs in the obvious fashion.\\

				$$ c_1, c_2 : E_{k_1^{b_1}, k_2^{b_2}} (k_3^{ G(b_1, b_2) } \Vert c_3) $$

				Where $E_{k_i, k_j}(m)$ is some encryption function taking the keys $k_i$ and $k_j$ and the plaintext $m$. Since the advent of AES-NI and the cheapness of using AES we will use AES with 128 bit keys to make this function. Suppose that $AES_k(m)$ denotes the AES encryption of the plaintext $m$ under the 128 bit key $k$. We define $E_K$ (and it's inverse $D_K$) as follows,

				$$ E_K(m) = AES_{k_1}( AES_{k_2}( ...AES_{k_n}(m) ...) ) \textnormal{, where } K = \{k_1, ..., k_n\}$$ 
				$$ D_K(m) = AES^{-1}_{k_n}( AES^{-1}_{k_{n-1}}( ...AES^{-1}_{k_1}(m) ...) ) \textnormal{, where } K = \{k_1, ..., k_n\}$$ 

				This is the inituitive extension of AES to multiple keys, chaining the encryption under all of the keys in a set order.\\

				Then $P_1$ (the builder of the circuit) sends this garbled version of the circuit to $P_2$ (the executor of the circuit). $P_1$ should send the garbling key for it's input bit ($k_1^{b_1}$), the full encrypted truth table and $c_1 = \pi(b_1)$. Then $P_2$ needs to get $k_2^{b_2} \Vert c_2$ from $P_2$ without revealing the value of $b_2$. This is done by an Oblivious Transfer where $P_1$ inputs $k_2^0$ and $k_2^1$ and $P_2$ inputs $b_2$. $P_2$ receives the output $k_2^{b_2} \Vert c_2$ from the OT and learns nothing about $k_2^{(b_2 \oplus 1)} $, $P_1$ gets no output and learns nothing about the value of $b_2$.\\

				$P_2$ can then look up the entry in the encrypted truth table indexed by $c_1$ and $c_2$ and decrypt it with using $k_1^{b_1}$ and  $k_2^{b_2}$. This will give $P_2$ a value for $k_3^{G(b_1, b_2)} \Vert c_3$. Then by using $\pi_3^{-1}$, $P_2$ can extract a value for $G(b_1, b_2)$.\\

				This can be extended to a full circuit, the input wires belonging to the circuits builder are hard coded and their garble keys and permutated values are sent to the executor. The values for the input wires belonging to the executor are obtained by the executor via Oblivious transfer with the builder. The executor is only given the permutations for the output wires, and therefore the intermediate values are protected.

			\subsubsection{Security of Yao Garbled Circuits}
				A naive implementation of a protocol using Yao Garbled Circuits only provides Semi-honest security. For a formal proof of Semi-honest security see \cite{ProofOfYaoSecurity}, we shall briefly give an intuitional explaination of why naive Yao Garbled Circuits are not secure in the presence of Malicious or Covert adversaries.\\
		
				Consider the case where the circuit building party is Malicious, at no point does the executing party verify that the garbled circuit provided by the builder actually computes the function the builder claims it does. This clearly breaks the Correctness requirement, but also opens up an attack on the Privacy and Independence of inputs.\\

				There is nothing to stop the Builder providing a circuit that outputs the input of the Executor in a way that the Builder can recover, as such the Privacy of the Executor's inputs fails in the presence of a Malicious Builder. Furthermore, the Builder could provide a circuit where its input is a function of the Executor's input, thus defeating the Independence of inputs requirement.


			\subsubsection{Security against Malicious and Covert Adversaries}
				Several extensions of Yao's original protocol have been proposed in order to achieve security against Malicious and Covert adversaries. Mostly depending on an appraoch dubbed ``cut and choose'' which provides statistical security (detects cheating with a certain probability). This relates to the old solution to dividing a cake fairly, one party cuts the cake in two, then the other party chooses a slice. In our case the Builder builds $s$ many garbled circuits and sends them to the Executor. Each of these circuits is chosen with probability $\frac{1}{2}$ as a check circuit that will be evaluated to test if they yield the correct result.\\

				If all check circuits pass then the Executor evaluates the remaining circuits as usual and returns the most common output. If one or more circuits produces an incorrect result the this indicates cheating, furthermore if any check circuits fail during evaluation this is also taken to indicate cheating. This means $s$ acts as a security parameter and the probability of detecting cheating is expressed in terms of $s$. For example the protocol proposed in \cite{Lindell_CnC_2014} detects cheating with probability $2^s$.\\

				This Cut and Choose approach creates additional problems to be solved. For example whilst evaluating the many circuits we must now also ensure that both parties' inputs are consistent (same inputs to each circuit) else they might be able learn many outputs, each revealing something they should not have been able to discover.\\

				In \cite{LindellAndPinkas2007} the example is given of computing the Inner product of two binary strings, in this situation the Executing party could give many different inputs each with a single bit set to 1. The output of the circuit would then give the Executor the value of the Builder's input bit corresponding to the high bit in the Executor's input.\\

				Furthermore, when detecting cheating in a check circuit the Executor cannot simply abort as this open them up to an attack where circuits are crafted to fail when the Executor's input fulfils a condition, leaking information about the Executor's input.


		\subsection{Oblivious Transfer} \label{OT_Intro}	
			\subsubsection{Introduction to Oblivious Transfer}
				Oblivious Transfers protocols allow for one party(called the receiver) to get exactly one out-of two (and can be extended to out-of-n) values from another party (called the Sender). The receiver is oblivious to the other value(s), and the Sender is oblivious to which value the receiver received.\\

				Oblivious Transfers (OTs) were first suggested by Rabin in \cite{Rabin81}. We formally define the functionality of a 1-out-of-2 OT protocol in Figure \ref{fig:OTformalDef}. Oblivious Transfers are vital to Yao Garbled Circuits, used to give the circuit executor the garble keys representing its inputs, without telling the circuit builder what those inputs were.\\

				\begin{figure}[!htb]
					\centering
					\begin{minipage}{0.45\textwidth}
						\centering
						\textbf{Receiver}\\
						Inputs : $b \in \{0, 1\}$\\
						Outputs : $X_b$\\
					\end{minipage}
					\begin{minipage}{0.45\textwidth}
						\centering
						\textbf{Sender}\\
						Inputs : $X_1$, $X_2$\\
						Outputs : $\emptyset$\\
					\end{minipage}

					\caption{ Formal definition of the functionality of a one-out-of-two OT protocol.\label{fig:OTformalDef}}
				\end{figure}
      
				The security of Oblivious Transfers is defined in a similar way to that of SMC, the focus is on Semi-honest(passive) and Malicious(active) adversaries. Security against these adversaries is usually either computational or statistical.\\

				A protocol is considered secure with regards to Semi-honest adversaries if neither a Semi-honest adversary in the sender role cannot learn anything about which value the receiver requested, nor can a Semi-honest adversary in the role of the Receiver learn anything about values other than the one it requested. The protocol being secure against Malicious adversaries is defined by the obvious extension of the Semi-honest case.\\

				Usually proof of the security of an OT is carried out via \emph{simulation}, proving that the protocol under consideration is equivalent to the ideal model with a trusted third party taking both parties inputs and sending the proper outputs to each party.

			\subsubsection{Even-Goldreich-Lempel SH OT Protocol}
				Here we look at a Semi-honest protocol for 1-out-of-2 case based on asymmetric encryption, taken from \cite{PinkasSlides2014} which is itself based on the work done in \cite{EvenEtAl85}. See Figure \ref{fig:EvenGoldreichLempel} for the abstracted protocol.\\

				\begin{figure}
					\begin{tabular}[!htb]{p{6cm} p{6cm}}
						\textbf{Receiver} & \textbf{Sender}\\
						Inputs : $b \in \{0, 1\}$ & Inputs : $X_1$, $X_2$\\
						Outputs : $X_b$ & Outputs : $\emptyset$\\
					\end{tabular}

					\begin{itemize}
						\setlength{\itemsep}{0.5pt}
						\setlength{\parskip}{0pt}
						\setlength{\parsep}{0pt}

						\item \textbf{Receiver:} Generates a public/private key pair $(E, D)$, where E is the public key.\\
						\item \textbf{Receiver:} Sets $PK_b = E$, choose $PK_{1-b}$ at random from the same distribution as the public keys. Send $PK_0$ and $PK_1$ to the Sender.\\
						\item \textbf{Sender:} Encrypt $X_0$ using $PK_0$ as $C_0$ and encrypt $X_1$ using $PK_1$ as $C_1$. Send $C_0$ and $C_1$ to the Receiver.\\
						\item \textbf{Receiver:} Receives $C_0$ and $C_1$, then decrypt $C_b$ using $D$. Output this decrypted value.
					\end{itemize}

					\caption{The abstracted Even-Goldreich-Lempel protocol. \label{fig:EvenGoldreichLempel}}
				\end{figure}

				This protocol assumes that it is possible to generate randomly in the \emph{form} the public key without revealing information about the private key. Given this assumption this protocol is clearly only Semi-honest as the Sender must trust the Receiver that $PK_{1-b}$ was indeed generated randomly and that the Receiver did not generate a decryption key.\\

				So long as a both parties are following the protocol, neither party can garner more information than they are supposed to be allowed to. Whilst this on its own is of limited real world application (we require a certain level of trust in someone we explicitly don't trust) some work has focused on taking Semi-honest protocols and extending them to be secure in the presence of Malicious adversaries via techniques like coin-flipping and zero knowledge proofs.
 
		\subsection{Oblivious Transfer using dual-mode cryptosystem} \label{dualModeCryptoOT}
			The basis of the Oblivious transfer protocol we shall be using comes from \cite{PVW_OT_2008}, in particular we shall be using the realisation of the dual-mode cryptosystem based on Decisional Diffie-Hellman problem. Whilst I shall not go into depth on this protocol we shall give a broad overview of the dual-mode cryptosystem.\\

			The Dual-mode cryptosystem has two modes (\emph{messy} and \emph{decryption}), selected during the setup, a Common Reference String(CRS) is also generated at setup. The Receiver uses their input bit and the CRS to generate  a ``base'' public key and private key. It should be noted that the security of their protocol depends on a trusted setup of the CRS. This is however, at least to Peikert et al., a reasonable assumption.\\

			The public key is then sent to the Sender, who computes two public keys derived from the base public key and the CRS. Each of the Sender's input values are then encrypted with one of these keys and the resulting ciphertexts sent to the Receiver. Finally the Receiver uses its private key in order to decrypt the ciphertext relating to its input bit.\\

			The properties of the dual-mode cryptosystem ensure that the Receiver can only decrypt one of the values. When generating the ``base'' keys the key generation takes as a parameter a decryptable branch ($\sigma \in \{0, 1\}$), and the resulting private key is associated with that branch. When encrypting using the public key we specify a branch (say $b \in \{0, 1\}$) on which to encrypt, the resulting ciphertext can only be decrypted if $\sigma = b$.\\

 			Further properties of the cryptosystem are that the first outputs of the setup of each modes are indistinguishable from one another, that use of a trapdoor value for the messy mode will reveal the encryptable branch of a public key and that a trapdoor value for the decryptable mode will allow the generation of keys decryptable on both branches.


		\subsection{Progress so far}
			So far (02/02/2015) we have implemented the naive Yao garbled circuit (with the Free XOR optimisation) using \ref{dualModeCryptoOT} for the Oblivious Transfer. We use the circuits provided on Prof. Smart's website \cite{NigelCircuits} and use the AES non-expanded circuit provided as our main benchmark. Running on the University of Bristol Cryptography group test machines Diffie and Hellman, my implementation can perform a single AES-128 block encryption in $\sim 1.3$ seconds. An approximate breakdown of which parts of the protocol are taking how long is provided in Figure \ref{fig:TimesBreakdown}.\\


	
			\begin{figure}[h!]
				\centering
				\begin{tabular}[!htb]{r | c | c}
					\textbf{Part of Implementation} & \textbf{CPU time(seconds)} & \textbf{Wall clock time(seconds)}\\
					\hline
					\textbf{Building party} & & \\
					Building Circuit & $0.180$ & $0.187$\\
					Sending circuit & $0.01$ & $0.043$\\
					OT &  $0.668$ & $0.897$\\

					\hline
					\textbf{Executing party} & & \\
					Receiving Circuit &  $0.02$ &  $0.059$\\
					OT & $0.316$ & $0.965$ \\
					Circuit Evaluation & $0.016$ & $0.017$\\
					Circuit Evaluation & $0.056$ & $0.056$\\
					without Free XOR & & \\

				\end{tabular}
				\caption{An informal breakdown of the timing of my implementation so far, running between Diffie and Hellman averaged over five runs. All times given in seconds. \label{fig:TimesBreakdown}}
			\end{figure}

			Most of this time is spent on the Oblivious Transfers, moreover the computational workload seems to be heavily weighted towards the Building party. This is particulary interesting as it shows the key factor of the time taken for the computation is the number of inputs the Executor has. This is not a suprising result as many have commented that in recent years the primary bottleneck has been shifting towards the Obilvious Transfer with optimisations like Free XOR evaluation and AES-NI. Another interesting results is that 



	\begin{thebibliography}{5}
		\bibitem{Lindell_CnC_2014}
			Fast Cut-and-Choose Based Protocols for Malicious and Covert Adversaries,\\
			Lindell (2014).

		\bibitem{SugarBeets}
			Multiparty Computation Goes Live,\\
			Peter Bogetoft et al. (2008)

		\bibitem{ProofOfYaoSecurity}
			A Proof of Security of Yao’s Protocol for Two-Party Computation,\\
			Lindell and Pinkas (2006).
    
		\bibitem{Rabin81}
			How to Exchange Secrets with Oblivious Transfer,\\
			Michael O. Rabin (1981)

		\bibitem{PinkasSlides2014}
			Secure Computation Lecture Series,\\
			Lecture 5 - Oblivious Transfer,\\
			Pinkas (2014)

		\bibitem{EvenEtAl85}
			A randomized protocol for signing contracts,\\
			Even, Goldreich and Lempel (1985)

		\bibitem{PVW_OT_2008}
 			A Framework for Efficient and Composable Oblivious Transfer,\\
			Peikert, Vaikuntanathan and Waters (2008)

		\bibitem{LindellAndPinkas2007}
			An Efficient Protocol for Secure Two-Party Compuation in the Presence of Malicious Adversaries,
			Lindell and Pinkas (2007).

		\bibitem{YaoOriginal}
			How to Generate and Exchange Secrets,
			A. Yao (1986).

		\bibitem{NigelCircuits}
			http://www.cs.bris.ac.uk/Research/CryptographySecurity/MPC/,
			Tillich and Smart. 

	\end{thebibliography}


\end{document}


